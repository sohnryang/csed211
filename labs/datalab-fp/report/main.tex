\documentclass{scrartcl}
\usepackage[margin=3cm]{geometry}
\usepackage{amsmath}
\usepackage{amssymb}
\usepackage{amsthm}
\usepackage{blindtext}
\usepackage{datetime}
\usepackage{fontspec}
\usepackage{float}
\usepackage{graphicx}
\usepackage{kotex}
\usepackage[lighttt]{lmodern}
\usepackage{listings}
\usepackage{mathrsfs}
\usepackage{mathtools}
\usepackage{pgf,tikz,pgfplots}

\pgfplotsset{compat=1.15}
\usetikzlibrary{arrows}
\newtheorem{theorem}{Theorem}

\lstset{
  numbers=none, frame=single, showspaces=false,
  showstringspaces=false, showtabs=false, breaklines=true, showlines=true,
  breakatwhitespace=true, basicstyle=\ttfamily, keywordstyle=\bfseries, basewidth=0.5em
}

\setmainhangulfont{Noto Serif CJK KR}[
  UprightFont=* Light, BoldFont=* Bold,
  Script=Hangul, Language=Korean, AutoFakeSlant,
]
\setsanshangulfont{Noto Sans CJK KR}[
  UprightFont=* DemiLight, BoldFont=* Medium,
  Script=Hangul, Language=Korean
]
\setmathhangulfont{Noto Sans CJK KR}[
  SizeFeatures={
    {Size=-6,  Font=* Medium},
    {Size=6-9, Font=*},
    {Size=9-,  Font=* DemiLight},
  },
  Script=Hangul, Language=Korean
]
\title{CSED211: Lab 2}
\author{손량(20220323)}
\date{Last compiled on: \today, \currenttime}

\newcommand{\un}[1]{\ensuremath{\ \mathrm{#1}}}

\begin{document}
\maketitle

\section{개요}
몇 가지 간단한 정수 관련 함수와 IEEE~754 표준에 따른 floating point number 관련
함수를 구현해 본다.

\section{코드 설명}

\subsection{\texttt{negate} -- Negate Signed Integer}
Two's complement 방식의 정의대로 계산하면 된다. 주어진 숫자의 bitwise not에 1을
더하였다.

\subsection{\texttt{isLess} -- Compare Signed Integers}
\texttt{x}가 \texttt{y}보다 작은 경우는 \texttt{x}가 음수, \texttt{y}가 0
이상이거나 둘의 부호가 같고 \texttt{x - y}가 음수인 경우임에 주목하여 코드를
작성하였다.

\subsection{\texttt{float\_abs} -- Absolute Value of Float}
우선 MSB만 0이고 나머지 비트는 1인 32비트 상수를 \texttt{sign\_mask}에 저장한
다음, 이를 \texttt{uf}와 bitwise and하여 결과값 \texttt{res}를 얻었다.
\texttt{uf}가 \texttt{NaN}일 경우 그대로 반환해야 하기 때문에, \texttt{NaN}
판정을 위해 \texttt{nan\_mask} 변수를 사용하였다. 이 변수에는
\texttt{Infinity}에 해당하는 값이 저장되어 있는데, \texttt{nan\_mask}와
\texttt{uf}의 bitwise and 결괏값이 \texttt{nan\_mask}와 같은 경우,
\texttt{uf}는 \texttt{Infinity}, \texttt{-Infinity}, \texttt{NaN} 셋 중 하나일
것이다. \texttt{res}가 \texttt{Infinity}가 아닌 경우, \texttt{uf}는
\texttt{NaN}이므로 \texttt{uf} 그대로를 반환한다. 아닌 경우에는 \texttt{res}에
담긴 결괏값이 정확함을 알 수 있고, \texttt{res}를 반환한다.

\subsection{\texttt{float\_twice} -- Double a Float}
우선 IEEE~754 floating point의 exponent 부분에 해당하는 비트만 1인 상수를
\texttt{exp\_mask}에 저장하고, \texttt{uf}에 저장해 exponent 부분을 가져온다.
만약 exponent에 해당하는 비트들이 모두 1이라면 이미 \texttt{Infinity}나
\texttt{NaN}이므로, \texttt{uf}를 그대로 반환한다. 그렇지 않은 경우에는
exponent에 해당하는 값을 1 증가시킨 후 \texttt{uf}의 exponent 부분을 덮어씌운
숫자를 반환한다.

\subsection{\texttt{float\_i2f} -- Signed Integer to Float}
주어진 정수 \texttt{x}의 크기에 따라 동작이 달라진다. \texttt{x}의 leading
zero를 제외한 부분의 길이를 \texttt{bitlen}에 저장하고, \texttt{bitlen}이
24보다 작은 경우에는 mantissa에 적절히 \texttt{x}의 절댓값을 left shift하여
저장하고, \texttt{bitlen}이 24인 경우에는 절댓값 그대로를 mantissa에 저장한다.
만약 \texttt{bitlen}이 24보다 큰 경우에는 round to even 규칙에 맞추어
rounding을 수행하여 mantissa를 설정한다. 최종 결과는 \texttt{x}의 부호에 맞게
sign bit를 설정하고, exponent와 mantissa를 표준에 맞게 합쳐 얻는다.

\subsection{\texttt{float\_f2i} -- Float to Signed Integer}
\texttt{uf}의 sign bit를 right shift로 얻어 음수인지 판정하여
\texttt{is\_negative}에 저장한다. \texttt{uf}의 exponent 부분과 mantissa 부분을
각각 \texttt{exp}, \texttt{mant}에 저장한 뒤, \texttt{exp}에 따라 변환을 다르게
수행한다. \texttt{exp}가 0 미만인 경우, \texttt{uf}는 절댓값이 1보다 작은 수를
나타내므로 결괏값은 0이 된다. \texttt{exp}가 31이고 \texttt{uf}가 음수이며
mantissa가 0인 경우에는 결괏값은 \texttt{INT\_MIN}이다. 그 이와에
\texttt{exp}가 31 이상인 경우에는 범위 밖으로 나간 것으로 간주하였다.
\texttt{exp}가 이외의 값을 가지는 경우에는 \texttt{mant}를 적당한 값만큼
shift하여 결괏값을 얻었다.

\end{document}
% vim: textwidth=79
